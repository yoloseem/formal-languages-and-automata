\documentclass[]{book}

%These tell TeX which packages to use.
\usepackage{array,epsfig}
\usepackage{amsmath}
\usepackage{amsfonts}
\usepackage{amssymb}
\usepackage{amsxtra}
\usepackage{amsthm}
\usepackage{mathrsfs}
\usepackage{color}
\usepackage{lipsum}
\usepackage{adjustbox}

%Here I define some theorem styles and shortcut commands for symbols I use often
\theoremstyle{definition}
\newtheorem{defn}{Definition}
\newtheorem{innercustomtheorem}{Theorem}
\newenvironment{thm}[1]
  {\renewcommand\theinnercustomtheorem{#1}\innercustomtheorem}
  {\endinnercustomtheorem}
\newtheorem{cor}{Corollary}
\newtheorem*{rmk}{Remark}
\newtheorem{lem}{Lemma}
\newtheorem*{joke}{Joke}
\newtheorem{innercustomexample}{Example}
\newenvironment{ex}[1]
  {\renewcommand\theinnercustomexample{#1}\innercustomexample}
  {\endinnercustomexample}
\newtheorem*{soln}{Solution}
\newtheorem{prop}{Proposition}

\renewcommand\qedsymbol{Q.E.D.}

%Pagination stuff.
\setlength{\topmargin}{-.3 in}
\setlength{\oddsidemargin}{0in}
\setlength{\evensidemargin}{0in}
\setlength{\textheight}{9.in}
\setlength{\textwidth}{6.5in}
\pagestyle{empty}

\begin{document}

\begin{center}
{\Large Automata and Theory of Computation} \par
(Book: Formal Languages and Automata 5th Edition) \par
\textbf{Hyunjun Kim}
\end{center}

\subsection*{Exercises for Section 4.1: Closure Properties of Regular Languages}

\begin{enumerate}
\item
    Fill in the details of the constructive proof of closure under intersection in Theorem~4.1.
\begin{proof}
    Let $L_1 = L(M_1)$ and $L_2 = L(M_2)$ such that $M_1 = (Q, \Sigma, \delta_0, q_0, F_1)$ and
    $M_2 = (P, \Sigma, \delta_1, p_0, F_2)$ are dfa's. We construct from $M_1$ and $M_2$
    a combined automaton $\hat{M}$, whose state set $\hat{Q} = Q \times P$ consists of
    pairs $(p_i, q_j)$, and whose transition function $\hat{\delta}$ is such that
    $\hat{M}$ is in state $(p_i, q_j)$ whenever $M_1$ is in state $q_i$ and $M_2$ is
    in state $p_j$. This is achieved by taking 
        $\hat{\delta}((q_i, p_j), a) = (q_k, p_l)$
    whenever
        $$\delta_1(q_i, a) = q_k$$
    and
        $$\delta_2(p_j, a) = p_l.$$
    $\hat{F}$ is defined as the set of all $(q_i, p_j)$, such that $q_i \in F_1$,
    and $p_j \in F_2$. Then it shows $w \in L_1 \cap L_2$ if and only if it is accepted by $\hat{M}$.
    Consequently $L_1 \cap L_2$ is regular.
\end{proof}

\item
    Use the construction in Theorem~4.1 to find nfa's that accept
    \begin{enumerate}
        \item $L((a + b )a^\ast) \cap L(baa^\ast)$,
        \item $L(ab^\ast a^\ast) \cap L(a^\ast b^\ast a)$.
    \end{enumerate}
\begin{soln}
    \begin{enumerate}
        \item A dfa for $L((a + b)a^\ast)$ is given by
                $$\delta(q_0, a) = q_1, \delta(q_0, b) = q_1, \delta(q_1, a) = q_1,
                \delta(q_1, b) = q_t,$$
            with $q_t$ a trap state and final state $q_1$. A dfa for $L(baa^\ast)$ is given by
                $$\delta(p_0, a) = p_t, \delta(p_0, b) = p_1, \delta(p_1, a) = p_2,
                \delta(p_1, b) = p_t, \delta(p_2, a) = p_2, \delta(p_2, b) = p_t,$$
            with $p_t$ a trap state and final state $p_2$. From this we find
                $$\delta((q_0, p_0), a) = (q_1, p_t), \delta((q_0, p_0), b) = (q_1, p_1),
                \delta((q_1, p_1), a) = (q_1, p_2),$$
            etc. When we complete this construction, we see that the only final state is
            $(q_1, p_2)$ and that $L((a + b)a^\ast) \cap L(baa^\ast) = L(baa^\ast)$.
        \item A dfa for $L(ab^\ast a^\ast)$ is given by
                $$\delta(q_0, a) = q_1, \delta(q_0, b) = q_t,$$
                $$\delta(q_1, a) = q_2, \delta(q_1, b) = q_1,$$
                $$\delta(q_2, a) = q_2, \delta(q_2, b) = q_t,$$
            with $q_t$ a trap state and final states $q_1$ and $q_2$. Likewise, a dfa for
            $L(a^\ast b^\ast a)$ is given by
                $$\delta(p_0, a) = p_1, \delta(p_0, b) = p_3,$$
                $$\delta(p_1, a) = p_2, \delta(p_1, b) = p_3,$$
                $$\delta(p_2, a) = p_2, \delta(p_2, b) = p_3,$$
                $$\delta(p_3, a) = p_4, \delta(p_3, b) = p_3,$$
                $$\delta(p_4, a) = p_t, \delta(p_4, b) = p_t,$$
            with $p_t$ a trap state and the three final states $p_1$, $p_2$, and $p_4$.
            From this we construct new nfa below
                $$\delta((q_0, p_0), a) = (q_1, p_1), \delta((q_0, p_0), b) = (q_t, p_3),$$
                $$\delta((q_1, p_1), a) = (q_2, p_2), \delta((q_1, p_1), b) = (q_1, p_3),$$
                $$\delta((q_1, p_3), a) = (q_2, p_4), \delta((q_1, p_3), b) = (q_1, p_3),$$
                $$\delta((q_2, p_2), a) = (q_2, p_2), \delta((q_2, p_2), b) = (q_t, p_3),$$
                $$\delta((q_2, p_4), a) = (q_2, p_t), \delta((q_2, p_4), b) = (q_t, p_t).$$
            From this construction,we see that the final states are
            $(q_1, p_1)$, $(q_2, p_2)$, and $(q_2, p_4)$.
            And that $L(ab^\ast a^\ast) \cap L(a^\ast b^\ast a) = L(a(a^\ast + b^\ast a))$.
    \end{enumerate}
\end{soln}

    \begin{ex}{4.1}
        Show that the family of regular languages is closed under difference.

        In other words, we want to show that if $L_1$ and $L_2$ are regular,
        then $L_1 - L_2$ is necessarily regular also. \par
        The needed set identity is immediately obvious from the definition of a set
        difference, namely $$L_1 - L_2 = L_1 \cap \overline{L_2}.$$
        The fact that $L_2$ is regular implies that $\overline{L_2}$ is also regular.
        Then, because of the closure of regular languages under intersection,
        we know that $L_1 \cap \overline{L_2}$ is regular, and the argument is complete.
    \end{ex}
    
\item 
    In Example~4.1 we showed closure under difference for regular languages,
    but the proof was nonconstructive. Provide a constructive argument for this result,
    following the approach used in the argument for intersection in Theorem~4.1.
    
    \begin{proof}
        Let $L_1 = L(M_1) and L_2 = L(M_2)$ such that $M_1 = (Q, \Sigma, \delta_0, q_0, F_1)$ and
        $M_2 = (P, \Sigma, \delta_1, p_0, F_2)$ are dfa's. We construct from $M_1$ and $M_2$
        a combined automaton $\hat{M}$, whose state set $\hat{Q} = Q \times P$ consists of
        pairs $(p_i, q_j)$ and whose transition function $\hat{\delta}$ is such that $\hat{M}$ is
        in state $(p_i, q_j)$ whenever $M_1$ is in state $q_i$ and $M_2$ is in state $p_j$.
        This is achieved by taking $\hat{\delta}((q_i, p_j), a) = (q_k, p_l)$ whenever
            $$\delta_1(q_i, a) = q_k$$ and
            $$\delta_2(p_j, a) = p_l.$$
        $\hat{F}$ is defined as set of all $(q_i, p_j)$, such that $q_i \in F_1$, and
        $p_j \notin F_2$. Then it shows $w \in L_1 - L_2$ if and only if it is accepted by
        $\hat{M}$. Consequently $L1 - L2$ is regular.
    \end{proof}

\begin{thm}{4.3}
    Let $h$ be a homomorphism. If $L$ is a regular language, then its homomorphic image
    $h(L)$ is also regular. The family of regular languages is therefore closed under
    arbitrary homomorphisms.
    
    \begin{proof}
        Let $L$ be a regular language denoted by some regular expression $r$. We find
        $h(r)$ by substituting $h(a)$ for each symbol $a \in \Sigma$ of $r$. It can be shown
        directly by an appeal to the definition of a regular expression that the result is a
        regular expression. It is equally easy to see that the resulting expression denotes
        $h(L)$. All we need to do is to show that for every $w \in L(r)$, the corresponding
        $h(w)$ is in $L(h(r))$ and conversely that for every $v$ in $L(h(r))$ there is a $w$
        in $L$, such that $v = h(w)$. Leaving details as an exercise, we claim that $h(L)$ is
        regular.
    \end{proof}
\end{thm}

\item
    In the proof of Theorem~4.3, show that $h(r)$ is a regular expression. Then show that
    $h(r)$ denotes $h(L)$.

    \begin{proof}
        A regular expression for $h(r)$ can be obtained by simply applying $h(a)$ for
        each symbol $a \in \Sigma$ of $r$. Now, we have to prove that the set of strings
        accepted by $h(r)$ is a subset of those accepted by $h(L)$, and the set of strings
        accepted by $h(L)$ is a subset of those accepted by $h(r)$, thus proving they are
        equivalent.
        
        $h(r) \subseteq h(L)$: All strings accepted by $h(r)$ must first follow the original
        rules of $r$, but each symbol $a$ in $\Sigma$ of $r$ has been substituted with
        $h(a)$ and thus each string accepted by $h(r)$ must be accepted by $h(L)$.
        
        $h(L) \subseteq h(r)$: All strings accepted by $h(L)$ must first follow the regular
        expression $r$, but rather than use the old symbols, will have them put through $h$
        first. Since this is the definition of $h(r)$, it is easy to see.
    \end{proof}
    
\item
    Show that the family of regular languages is closed under finite union and intersection,
    that is, if $L_1$, $L_2$, ..., $L_n$ are regular, then
        $$L_U = \displaystyle\bigcup_{i=\{1, 2, ..., n\}} L_i$$ and
        $$L_I = \displaystyle\bigcap_{i=\{1, 2, ..., n\}} L_i$$ are also regular.
    
    \begin{proof}
        Induction on $n$ to show that if $L_1$, $L_2$, ..., $L_n$ are regular,
        $L_U = \cup_{i=\{1, 2, ...n\}} L_i$ is also regular. \par
        \underline{Base case}: If $n = 1$, the expanded right side is $L_1$ and it is regular.
        So the argument holds when $n = 1$, now we have a base case. \par
        \underline{Inductive step}: Let $k \geq 1$ be an arbitrary natural number.\\
        Let assume the induction hypothesis: the argument holds for all values of $n$ up to
        some $k$. \\
        Then, with $n = n + 1$, $L_U = \displaystyle\bigcup_{i=\{1, 2, .., n+1\}} L_i$
        $= \displaystyle\bigcup_{i=\{1, 2, .., n\}} L_i \cup L_{n+1}$.
        From the latest formula, the left operand of $\cup$ is regular by inductive hypothesis,
        and the right operand is also regular immediately by the argument.
        Since the family of regular languages are closed under union, now we see
        that the argument holds for $n = n + 1$, consequently holds for any arbitrary $n$,
        $n \geq 1$.
    \end{proof}
    
    Showing that the family of regular languages is closed under finite intersection will
    taken easy as above.
    
\item
    The \textit{symmetric difference} of two sets $S_1$ and $S_2$ is defined as
    \begin{center}
        $S_1 \ominus S_2 = \{x: x \in S_1$ or $x \in S_2$, but $x$ is not in both
        $S_1$ and $S_2\}$.
    \end{center}
    Show that the family of regular languages is closed under symmetric difference.
    
    \begin{proof}
        \begin{equation*}
        \begin{split}
            S_1 \ominus S_2 & = (S_1 \cup S_2) - (S_1 \cap S_2) \\
            & = (S_1 \cup S_2) \cap \overline{(S_1 \cap S_2)}
        \end{split}
        \end{equation*}
        We know the family of regular languages is closed under union,
        intersection, and complementation. Now we see that it is also closed
        under symmetric difference.
    \end{proof}
    
\end{enumerate}

\end{document}


